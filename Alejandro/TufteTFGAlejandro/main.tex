%%%%%%%%%%%%%%%%%%%%%%%%%%%%%%%%%%%%%%%%%
% Tufte-Style Book (Minimal Template)
% LaTeX Template
% Version 1.0 (5/1/13)
%
% This template has been downloaded from:
% http://www.LaTeXTemplates.com
%
% License:
% CC BY-NC-SA 3.0 (http://creativecommons.org/licenses/by-nc-sa/3.0/)
%
% IMPORTANT NOTE:
% In addition to running BibTeX to compile the reference list from the .bib
% file, you will need to run MakeIndex to compile the index at the end of the
% document.
%
%%%%%%%%%%%%%%%%%%%%%%%%%%%%%%%%%%%%%%%%%

%----------------------------------------------------------------------------------------
%	PACKAGES AND OTHER DOCUMENT CONFIGURATIONS
%----------------------------------------------------------------------------------------

\documentclass[a4paper,justified,titlepage.nobib]{tufte-book} 
%\documentclass[a4paper,sfsidenotes,notitlepage,justified,nobib]{tufte-book}
% Use the tufte-book class which in turn uses the tufte-common class

\hypersetup{colorlinks} % Comment this line if you don't wish to have colored links

%----------------------------------------------------------------------------------------
%	MY PACKAGES
%----------------------------------------------------------------------------------------
\usepackage[utf8]{inputenc} % Para caracteres especiales en español
\usepackage[spanish]{babel} % Configura el documento en español

%\usepackage{url}
\usepackage[nobiblatex]{xurl} %better than url
\usepackage{import}

\usepackage[inline]{enumitem}%modification enum

\usepackage[binary-units = true,range-phrase = -- , list-pair-separator = /,per-mode=symbol,per-symbol = p]{siunitx}
\DeclareSIUnit\dBi{\dB i}

\usepackage{fmtcount}%allows printing numbers as words in different languages

\usepackage{tikz} %For graphics
\usetikzlibrary{mindmap,calc,matrix,chains,positioning,decorations.pathreplacing,arrows}
%----------------------------------------------------------------------------------------

\usepackage{microtype} % Improves character and word spacing

\usepackage{lipsum} % Inserts dummy text

\usepackage{booktabs} % Better horizontal rules in tables

\usepackage{graphicx} % Needed to insert images into the document
\graphicspath{{Introduction/}{Chapter1/}{Chapter2/}{Chapter3/}{Chapter4/}} % Sets the default location of pictures
\setkeys{Gin}{width=\linewidth,totalheight=\textheight,keepaspectratio} % Improves figure scaling

\usepackage{fancyvrb} % Allows customization of verbatim environments
\fvset{fontsize=\normalsize} % The font size of all verbatim text can be changed here

\newcommand{\hangp}[1]{\makebox[0pt][r]{(}#1\makebox[0pt][l]{)}} % New command to create parentheses around text in tables which take up no horizontal space - this improves column spacing
\newcommand{\hangstar}{\makebox[0pt][l]{*}} % New command to create asterisks in tables which take up no horizontal space - this improves column spacing

\usepackage{xspace} % Used for printing a trailing space better than using a tilde (~) using the \xspace command

\newcommand{\monthyear}{\ifcase\month\or January\or February\or March\or April\or May\or June\or July\or August\or September\or October\or November\or December\fi\space\number\year} % A command to print the current month and year

%\newcommand{\openepigraph}[2]{ % This block sets up a command for printing an epigraph with 2 arguments - the quote and the author
%\begin{fullwidth}
%\sffamily\large
%\begin{doublespace}
%\noindent\allcaps{#1}\\ % The quote
%\noindent\allcaps{#2} % The author
%\end{doublespace}
%\end{fullwidth}
%}

\newcommand{\blankpage}{\newpage\hbox{}\thispagestyle{empty}\newpage} % Command to insert a blank page

\usepackage{makeidx} % Used to generate the index

\makeindex % Generate the index which is printed at the end of the document





%----------------------------------------------------------------------------------------
%	BOOK META-INFORMATION
%----------------------------------------------------------------------------------------

\makeatletter
\renewcommand{\maketitlepage}{%
\begingroup%
\setlength{\parindent}{0pt}
\newgeometry{left=2.4cm}

{\fontsize{24}{24}\selectfont\textit{\@author}\par}

\vspace{1.75in}{\fontsize{30}{40}\selectfont\@title\par}

\vspace{0.5in}{\fontsize{14}{14}\selectfont\textsf{\smallcaps{\@date}}\par}

\vspace{1in}{\fontsize{14}{14}\selectfont\textbf{Director:}\par}
{\fontsize{14}{14}\selectfont\textit{Martínez Cabeza de Vaca Alajarín, Juan de la Cruz}\par}

\vspace{0.5in}{\fontsize{14}{14}\selectfont\textbf{Codirectores:}\par}
{\fontsize{14}{14}\selectfont\textit{Zapata Pérez, Juan Francisco}\par}
{\fontsize{14}{14}\selectfont\textit{Arévalo García, Alicia}\par}

\vfill{\fontsize{14}{14}\selectfont\textit{\@publisher}\par}

\thispagestyle{empty}
\endgroup
}
\makeatother
\title[Herramientas IA en la Planificación Quirúrgica]{Herramientas de IA en la Planificación Quirúrgica: Segmentación y Clasificación Automatizada en Imágenes Médicas RM}
\date{Mayo 2025}
\author[Alejandro Martínez Guillermo{Alejandro Martínez Guillermo}]{Alejandro Martínez Guillermo}
\publisher{Depto. de Electrónica, Tecnología de Computadores y Proyectos}
\publisher{Universidad Politécnica de Cartagena}


%%
% If they're installed, use Bergamo and Chantilly from www.fontsite.com.
% They're clones of Bembo and Gill Sans, respectively.
% \IfFileExists{bergamo.sty}{\usepackage[osf]{bergamo}}{}% Bembo
% \IfFileExists{chantill.sty}{\usepackage{chantill}}{}% Gill Sans

%----------------------------------------------------------------------------------------
%	GLOSSARY (Abbreviations}
% Any abbreviation must be defined here
%----------------------------------------------------------------------------------------
\usepackage[nomain,automake,abbreviations]{glossaries-extra}
\makeglossaries
\setabbreviationstyle[acronym]{long-short}

\newacronym{UPCT}{UPCT}{Universidad Polit\'ecnica de Cartagena}
\newacronym{laser}{laser}{light amplification by stimulated
emission of radiation}
\newacronym{ANN}{ANN}{Artificial Neural Network}
\newacronym{CNN}{CNN}{Convolutional Neural Network}
\newacronym{DCNN}{DNN}{Deep Convolutional Neural Network}
\newacronym{DNN}{DNN}{Deep Neural Network}
\newacronym{ML}{ML}{Machine Learning}
\newacronym{CPU}{CPU}{Central Processing Unit}
\newacronym{FC}{FC}{Fully Connected (layer or network)}
\newacronym{FCN}{FCN}{Fully Convolutional Network}
\newacronym{GPU}{GPU}{Graphics Processing Unit}
\newacronym{IoU}{IoU}{Intersection over Union}
\newacronym{NMS}{NMS}{Non-maximum suppression}
\newacronym{R-CNN}{R-CNN}{Regions with convolutional neural network features} 
\newacronym{RoI}{RoI}{Region of Interest} 
\newacronym{RPN}{RPN}{Region Proposal Network} 
\newacronym{SSD}{SSD}{Single Shot MultiBox Detector}
\newacronym{SVM}{SVM}{Support Vector Machine}
\newacronym{NDT}{NDT}{Non-Destructive Testing}
\newacronym{RT}{RT}{Radiographic Testing}
\newacronym{AI}{AI}{Artificial Intelligence}
\newacronym{DARPA}{DARPA}{Defense Advanced Research Projects Agency}
\newacronym{SEAC}{SEAC}{Standards Eastern Automatic Computer}
\newacronym{NIST}{NIST}{National Institute of Standards and Technology}
\newacronym{ANFIS}{ANFIS}{Adaptive-Network-Based Fuzzy Inference System}

%\renewcommand{\HUGE{\fontsize{60}{60}\selectfont}}}
\titleformat{\chapter}%
  [display]% shape
  {\relax\ifthenelse{\NOT\boolean{@tufte@symmetric}}{\begin{flushleft}}{}}% format applied to label+text
%   {\itshape\LARGE\chaptertitlename~\thechapter}% label
  {\itshape\fontsize{60}{60}\selectfont\thechapter}% label
  {0pt}% horizontal separation between label and title body
  {\Huge\rmfamily\itshape}% before the title body
  [\ifthenelse{\NOT\boolean{@tufte@symmetric}}{\end{flushleft}}{}]% after the title body
\makeatother
%\setcounter{secnumdepth}{1}

\setcounter{secnumdepth}{2}%Supposed "Switch" to turn numbering on to subsection or level 2
%Found as a response here: https://groups.google.com/forum/#!topic/tufte-latex/s5aAqdvDSp

\let\cite\citep


\begin{document}

%----------------------------------------------------------------------------------------
%	EPIGRAPH
%----------------------------------------------------------------------------------------

%\thispagestyle{empty}
%\openepigraph{Quotation 1}{Author, {\itshape Source}}
%\vfill
%\openepigraph{Quotation 2}{Author}
%\vfill
%\openepigraph{Quotation 3}{Author}

%----------------------------------------------------------------------------------------

%\maketitle % Print the title page
\frontmatter
\maketitle

%----------------------------------------------------------------------------------------
%	COPYRIGHT PAGE
%----------------------------------------------------------------------------------------

\newpage
\begin{fullwidth}
~\vfill
\thispagestyle{empty}
\setlength{\parindent}{0pt}
\setlength{\parskip}{\baselineskip}
Copyright \copyright\ \the\year\ \thanklessauthor

\par\smallcaps{Trabajo Fín de Grado \thanklessauthor}
\par\smallcaps{Presentado en \thanklesspublisher}

%\par\url{http://www.bookwebsite.com}

%\par License information.\index{license}

\par\textit{First printing, \monthyear}
\end{fullwidth}

%----------------------------------------------------------------------------------------
%	ACKNOWLEDGMENT
%----------------------------------------------------------------------------------------
%\subimport{../}{Acknowlegments}
\chapter*{Agradecimientos}
\label{ack}
 Quiero agradecer a mi familia por su apoyo incondicional, a mis amigos por su compañía y a mi tutor por su paciencia y dedicación.
%\cleardoublepage


\cleardoublepage

%----------------------------------------------------------------------------------------
%	ABSTRACT
%----------------------------------------------------------------------------------------
%\subimport{../}{Abstract}
\chapter*{Resumen}
\label{cha:abs}

Resumen (te puede servir la ficha del proyecto donde hay que exponer los objetivos, metodología, resultados y conclusiones).

\cleardoublepage

\cleardoublepage

%\chapter*{Abstract}
\label{cha:abs}
Translation of abstract in spanish
 %In spanish
%\cleardoublepage
%----------------------------------------------------------------------------------------
%	CONTENTS
%----------------------------------------------------------------------------------------

\tableofcontents % Print the table of contents

%----------------------------------------------------------------------------------------

\listoffigures % Print a list of figures

%----------------------------------------------------------------------------------------

\listoftables % Print a list of tables

%----------------------------------------------------------------------------------------
%	DEDICATION PAGE
%----------------------------------------------------------------------------------------

\cleardoublepage
~\vfill
\begin{doublespace}
\noindent\fontsize{18}{22}\selectfont\itshape
\nohyphenation
Dedicado a mi familia y amigos.
\end{doublespace}
\vfill
\vfill

%----------------------------------------------------------------------------------------
%	INTRODUCTION
%----------------------------------------------------------------------------------------

\mainmatter

\subimport{./}{Introduction}
\cleardoublepage

%----------------------------------------------------------------------------------------
%	CHAPTER 2
%----------------------------------------------------------------------------------------

%\subimport{./}{chapter2}
%\cleardoublepage

%----------------------------------------------------------------------------------------
%	CHAPTER 3
%----------------------------------------------------------------------------------------

%\subimport{./}{chapter3}
%\cleardoublepage

%----------------------------------------------------------------------------------------
%	CHAPTER 4
%----------------------------------------------------------------------------------------

%\subimport{./}{chapter4}
%\cleardoublepage


%----------------------------------------------------------------------------------------
%	CHAPTER 5
%----------------------------------------------------------------------------------------

%----------------------------------------------------------------------------------------
%	CHAPTER 6
%----------------------------------------------------------------------------------------

%----------------------------------------------------------------------------------------
\backmatter
\printabbreviations

%----------------------------------------------------------------------------------------
%	BIBLIOGRAPHY
%----------------------------------------------------------------------------------------

\bibliography{deeplearning} % Use the bibliography.bib file for the bibliography
\bibliographystyle{plainnat} % Use the plainnat style of referencing

%----------------------------------------------------------------------------------------

\printindex % Print the index at the very end of the document

\end{document}

% Activate the following line by filling in the right side. If for example the name of the root file is Main.tex, write
% "...root = Main.tex" if the chapter file is in the same directory, and "...root = ../Main.tex" if the chapter is in a subdirectory.
 
%!TEX root =  ../main.tex

\tikzset{%
  every neuron/.style={
    circle,
    draw,
    minimum size=1cm
  },
  neuron missing/.style={
    draw=none, 
    scale=4,
    text height=0.333cm,
    execute at begin node=\color{black}$\vdots$
  },
}

\begin{tikzpicture}[x=1.5cm, y=1.5cm, >=stealth]

\foreach \m/\l [count=\y] in {1,2,3,missing,4}
  \node [every neuron/.try, neuron \m/.try] (input-\m) at (0,2.5-\y) {};

\foreach \m [count=\y] in {1,missing,2}
  \node [every neuron/.try, neuron \m/.try ] (hidden1-\m) at (2,2-\y*1.25) {};

\foreach \m [count=\y] in {1,missing,2}
\node [every neuron/.try, neuron \m/.try ] (hidden1-\m) at (2,2-\y*1.25) {};

\foreach \m [count=\y] in {1,missing,2}
  \node [every neuron/.try, neuron \m/.try ] (hidden2-\m) at (5,2-\y*1.25) {};

\foreach \m [count=\y] in {1,missing,2}
  \node [every neuron/.try, neuron \m/.try ] (output-\m) at (7,1.5-\y) {};

\foreach \l [count=\i] in {1,2,3,n}
  \draw [<-] (input-\i) -- ++(-1,0)
    node [above, midway] {$I_\l$};

\foreach \l [count=\i] in {1,n}
  \node [above] at (hidden1-\i.north) {$H_\l$};

\foreach \l [count=\i] in {1,n}
  \node [above] at (hidden2-\i.north) {$H_\l$};

\foreach \l [count=\i] in {1,n}
  \draw [->] (output-\i) -- ++(1,0)
    node [above, midway] {$O_\l$};

\foreach \i in {1,...,4}
  \foreach \j in {1,...,2}
    \draw [->] (input-\i) -- (hidden1-\j);

\foreach \i in {1,...,2}
  \foreach \j in {1,...,2}
    \draw [->] (hidden1-\i) -- (hidden2-\j);

    \foreach \i in {1,...,2}
    \foreach \j in {1,...,2}
    \draw [->] (hidden2-\i) -- (output-\j);

\node [align=center, above] at (0,2) {Input\\layer};
\node [align=center, above] at (2,2) {Hidden \\layer};
\node [align=center, above] at (5,2) {Hidden \\layer};
\node [align=center, above] at (7,2) {Output \\layer};

\node[fill=white,scale=4,inner xsep=0pt,inner ysep=5mm] at ($(hidden1-1)!.5!(hidden2-2)$) {$\dots$};

\end{tikzpicture}

%\def\layersep{60pt}
%
%\begin{tikzpicture}[shorten >=1pt,->,draw=black!50, node distance=\layersep]
%    \tikzstyle{every pin edge}=[<-,shorten <=1pt]
%    %\tikzstyle{every node}=[font=\tiny]
%    \tikzstyle{neuron}=[circle,fill=black!25,minimum size=17pt,inner sep=0pt]
%    \tikzstyle{input neuron}=[neuron, fill=green!50];
%    \tikzstyle{output neuron}=[neuron, fill=red!50];
%    \tikzstyle{hidden neuron}=[neuron, fill=blue!50];
%    \tikzstyle{annot} = [text width=2em, text centered]
%
%    % Draw the input layer nodes
%    \foreach \name / \y in {1,...,3}
%    % This is the same as writing \foreach \name / \y in {1/1,2/2,3/3,4/4}
%        \node[input neuron, pin=left: Input \#\y] (I-\name) at (0,-\y) {};
%		%\node[input neuron] (I-\name) at (0,-\y) {};
%    % Draw the hidden layer nodes
%    \foreach \name / \y in {1,...,4}
%        \path[yshift=0.5cm]
%            node[hidden neuron] (H-\name) at (\layersep,-\y cm) {};
%
%    % Draw the output layer node
%    \node[output neuron,pin={[pin edge={->}]right:Output}, right of=H-3] (O) at ($(H-2)!0.5!(H-3)$) {};
%%\node[output neuron,right of=H-3] (O) at ($(H-2)!0.5!(H-3)$) {};
%    % Connect every node in the input layer with every node in the
%    % hidden layer.
%    \foreach \source in {1,...,3}
%        \foreach \dest in {1,...,4}
%            \path (I-\source) edge (H-\dest);
%
%    % Connect every node in the hidden layer with the output layer
%    \foreach \source in {1,...,4}
%        \path (H-\source) edge (O);
%
%    % Annotate the layers
%    \node[annot,above of=H-1, node distance=1cm] (hl) { Hidden Layer};
%    \node[annot,left of=hl] { Input Layer};
%    \node[annot,right of=hl] { Output Layer};
%\end{tikzpicture}
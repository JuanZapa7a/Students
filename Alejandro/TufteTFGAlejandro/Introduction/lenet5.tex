%\noindent\begin{minipage}[c]{\linewidth}\centering
% \vspace{\baselineskip} \small 
%\mtotex{g}{quick60}
%1 row


\newcommand{\xangle}{0}
\newcommand{\yangle}{90}
\newcommand{\zangle}{220}

\newcommand{\xlength}{1}
\newcommand{\ylength}{1}
\newcommand{\zlength}{0.9}

\pgfmathsetmacro{\xx}{\xlength*cos(\xangle)}
\pgfmathsetmacro{\xy}{\xlength*sin(\xangle)}
\pgfmathsetmacro{\yx}{\ylength*cos(\yangle)}
\pgfmathsetmacro{\yy}{\ylength*sin(\yangle)}
\pgfmathsetmacro{\zx}{\zlength*cos(\zangle)}
\pgfmathsetmacro{\zy}{\zlength*sin(\zangle)}

%\def\space{.5} %space betwwen layer
%(0,0,0) coordinate (O);
 \begin{tikzpicture}
[   x={(\xx ,\xy )},
    y={(\yx ,\yy )},
    z={(\zx ,\zy )},
]
 
\newcommand{\networkLayer}[9]{
			\def\H{#1}% Height Y
			\def\W{#2}% Width XZ
			\def\D{#3} % Depth X
			\def\h{#4} % Height  Filter y 				
			\def\w{#5} % Width XZ Filter
			\def\s{#6} % s distance to next layer 
			\def\oh{#7}% N=1 or S=-1
			\def\ow{#8} %E=1 or W=-1
			\networkLayercontinued#9
			%\def\S1{#8} %
			%\def\S2{#9} %
			%\def\S3{#10} %
			% Draw the layer body.
}			
%NNSVG node[midway,below]{\tiny #3}  node[midway,right,xshift=-2]{\tiny #1}  node[midway,right]{\tiny #2} Tensor opacity Filter opacity Spacing between layers 

\newcommand{\networkLayercontinued}[6]{
% %Filter lateral face  
%\filldraw[opacity=0.8,fill=black!30] (.5*\w,-.5*\h,.5*\w) --++ (0,\h,0)  --++ (-\w,0,-\w) --++ (0,-\h,0) -- (.5*\w,-.5*\h,.5*\w);
%
%%Filter Front Face 
%  \filldraw[opacity=0.8,fill=black!30] (.5*\w,-.5*\h,.5*\w) --++ ( \D,0,0) --++ (0,\h,0) --++  ( -\D,0,0) --++   (0,-\h,0)node[opacity=1,midway,right]{\tiny #4} ;
%  
% %Filter Top Face 
%  \filldraw[opacity=0.8,fill=black!30] (.5*\w,.5*\h,.5*\w) --++ (-\w,0,-\w)node[opacity=1,midway,right]{\tiny #5}  --++ ( \D,0,0)  --++ (\w,0,\w)  --++ ( -\D,0,0); 
%
% \coordinate (A) at (+\D+.5*\w,-.5*\h,.5*\w);
% \coordinate (B) at (+\D+.5*\w,-.5*\h+\h,.5*\w);
% \coordinate (C) at (+\D+.5*\w-\w,-.5*\h+\h,.5*\w-\w);
% \coordinate (D) at (\D+\s,0,0);
% \draw (A) -- (D);
% \draw (B) -- (D);
% \draw (C) -- (D);

% Filter lateral face  
\filldraw[opacity=0.8,fill=black!20] (.25*\ow*\W+0.5*\w,\oh*.25*\H-0.5*\h,.25*\ow*\W+0.5*\w) --++ (0,\h,0)  --++ (-\w,0,-\w) --++ (0,-\h,0) -- (.25*\ow*\W+0.5*\w,\oh*.25*\H-0.5*\h,.25*\ow*\W+0.5*\w);

%Filter Front Face 
  \filldraw[opacity=0.8,fill=black!20] (.25*\ow*\W+0.5*\w,\oh*.25*\H-0.5*\h,.25*\ow*\W+0.5*\w) --++ ( \D,0,0) --++ (0,\h,0) --++  ( -\D,0,0) --++   (0,-\h,0)node[opacity=1,midway,right]{\tiny #4} ;
  
%Filter Top Face 
  \filldraw[opacity=0.8,fill=black!20] (.25*\ow*\W+0.5*\w,\oh*.25*\H-0.5*\h+\h,.25*\ow*\W+0.5*\w) --++ (-\w,0,-\w)node[opacity=1,midway,right]{\tiny #5}  --++ ( \D,0,0)  --++ (\w,0,\w)  --++ ( -\D,0,0); 
\coordinate (A) at (\D+.25*\ow*\W+0.5*\w,\oh*.25*\H-0.5*\h,.25*\ow*\W+0.5*\w);
 \coordinate (B) at (\D+.25*\ow*\W+0.5*\w,\oh*.25*\H-0.5*\h+\h,.25*\ow*\W+0.5*\w);
 \coordinate (C) at (\D+.25*\ow*\W+0.5*\w-\w,\oh*.25*\H-0.5*\h+\h,.25*\ow*\W+0.5*\w-\w);
 \coordinate (D) at (\D+.25*\ow*\W+0.5*\w+\s -.5*\w,\oh*.25*\H-0.5*\h,.25*\ow*\W+0.5*\w-.5*\w);
 \draw[densely dotted] (A) -- (D);
 \draw[densely dotted] (B) -- (D);
 \draw[densely dotted] (C) -- (D);
%(0,0,0) coordinate (o);

%Lateral Face
  \filldraw[opacity=0.5,fill=black!20] (.5*\W,-.5*\H,.5*\W) --++ (0,\H,0)  --++ (-\W,0,-\W) --++ (0,-\H,0) -- (.5*\W,-.5*\H,.5*\W);
 %Front Face 
  \filldraw[opacity=0.5,fill=black!20] (.5*\W,-.5*\H,.5*\W) --++ ( \D,0,0)node[opacity=1,midway,below,label=below:\tiny #6]{\tiny #3} --++ (0,\H,0) --++  ( -\D,0,0) --++   (0,-\H,0)node[opacity=1,midway,right]{\tiny #1} ;
  %Top Face 
  \filldraw[opacity=0.5,fill=black!20] (.5*\W,.5*\H,.5*\W) --++ (-\W,0,-\W)node[opacity=1,midway,right]{\tiny #2}  --++ ( \D,0,0)  --++ (\W,0,\W)  --++ ( -\D,0,0);



\tikzset{shift={(\D+\s,0,0)}};

}
% From left to right
% 1 Height Y, 2 Width XZ, 3 Depth X, 4 Height  Filter y, 5 Width XZ Filter, 6 s distance to next layer,7 N=1 or S=-1,8 E=1 or W=-1,  9 text H, 10 text W, 11 Text depth, 12 text w filter, 13 text height filter ,14 type of layer

\networkLayer{3.2}{3.2}{0.1}{0}{0}{1}{0}{0}{{32}{32}{1}{}{}{Image}}
\networkLayer{2.8}{2.8}{0.6}{0.5}{0.5}{0.8}{1}{-1}{{23}{23}{5}{5}{5}{Conv}}
\networkLayer{1.4}{1.4}{0.6}{0.2}{0.2}{0.5}{1}{-1}{{14}{14}{6}{2}{2}{Av. Pool}}
\networkLayer{1}{1}{1.6}{0.5}{0.5}{0.3}{1}{1}{{10}{10}{16}{5}{5}{Conv}}
\networkLayer{0.5}{0.5}{1.6}{0}{0}{0.3}{1}{1}{{5}{5}{16}{}{}{Av. Pool}}
\networkLayer{6.0}{0.1}{0.1}{0}{0}{1}{0}{0}{{120}{1}{1}{}{}{Fully Conn}}
\networkLayer{4.2}{0.1}{0.1}{0}{0}{1}{0}{0}{{84}{1}{1}{}{}{Fully Conn}}
\networkLayer{1}{0.1}{0.1}{0}{0}{0}{0}{0}{{10}{1}{1}{}{}{0-9 digits}}
%\networkLayer{5}{0.1}{0.01}{0}{0}{1}{0}{0}{{50}{1}{1}{}{}{}}
%\networkLayer{5}{0.1}{0.01}{0}{0}{1}{0}{0}{{50}{1}{1}{}{}{}}
%\networkLayer{5}{0.1}{0.01}{0}{0}{0}{0}{0}{{50}{1}{1}{}{}{}}



\end{tikzpicture}

%\captionof{figure}{.}
%\label{Symbolic_label52}
%\end{minipage}
